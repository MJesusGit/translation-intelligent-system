\documentclass[a4paper]{article}

\usepackage[pages=all, color=black, position={current page.south}, placement=bottom, scale=1, opacity=1, vshift=5mm]{background}
\SetBgContents{

}      % copyright

\usepackage[margin=1in]{geometry} % full-width

% AMS Packages
\usepackage{amsmath}
\usepackage{amsthm}
\usepackage{amssymb}

% Unicode
\usepackage[utf8]{inputenc}
\usepackage{hyperref}
\hypersetup{
	unicode,
%	colorlinks,
%	breaklinks,
%	urlcolor=cyan, 
%	linkcolor=blue, 
	pdfauthor={Author One, Author Two, Author Three},
	pdftitle={A simple article template},
	pdfsubject={A simple article template},
	pdfkeywords={article, template, simple},
	pdfproducer={LaTeX},
	pdfcreator={pdflatex}
}

% Vietnamese
%\usepackage{vntex}

% Natbib
\usepackage[sort&compress,numbers,square]{natbib}
\bibliographystyle{mplainnat}

% Theorem, Lemma, etc
\theoremstyle{plain}
\newtheorem{theorem}{Theorem}
\newtheorem{corollary}[theorem]{Corollary}
\newtheorem{lemma}[theorem]{Lemma}
\newtheorem{claim}{Claim}[theorem]
\newtheorem{axiom}[theorem]{Axiom}
\newtheorem{conjecture}[theorem]{Conjecture}
\newtheorem{fact}[theorem]{Fact}
\newtheorem{hypothesis}[theorem]{Hypothesis}
\newtheorem{assumption}[theorem]{Assumption}
\newtheorem{proposition}[theorem]{Proposition}
\newtheorem{criterion}[theorem]{Criterion}
\theoremstyle{definition}
\newtheorem{definition}[theorem]{Definition}
\newtheorem{example}[theorem]{Example}
\newtheorem{remark}[theorem]{Remark}
\newtheorem{problem}[theorem]{Problem}
\newtheorem{principle}[theorem]{Principle}

\usepackage{graphicx, color}
\graphicspath{{fig/}}

%\usepackage[linesnumbered,ruled,vlined,commentsnumbered]{algorithm2e} % use algorithm2e for typesetting algorithms
\usepackage{algorithm, algpseudocode} % use algorithm and algorithmicx for typesetting algorithms
\usepackage{mathrsfs} % for \mathscr command

\usepackage{lipsum}

% Author info
\pagebreak
\title{Translation}
\author{María Jesús Dueñas Recuero}

\date{

 \\ \texttt{MJesus.duenas1@alu.uclm.es}\\[2ex]%
%	\today
}

\begin{document}
	\maketitle
	\newpage
	\
	\tableofcontents
	\newpage
	\section{Introduction}
	\label{sec:intro}
	\begin{flushleft}
	Given the increase in globalisation, every day we are in contact with a greater number of cultures, which forces us, in a healthy way, to learn new languages.

	However, thanks to the development of technology and artificial intelligence, it is not necessarily necessary to study a language for years in order to understand it, given that translator websites and the translators themselves allow us to interact with it fluently.

	It is an opportunity to broaden our knowledge and understanding of other languages. As well as guaranteeing the correct use of the language for work and business, as they not only translate a language but also interpret emotions and feelings.
	
    \end{flushleft}
	
	


	
	
	\section{First aproaches}
	\label{sec:examples}
	\begin{flushleft}
	De toda la vida se han llevado acabo traducciones.A dia de hoy la traduccion automatica ya no es una fantasaia de ciencia ficcinon, lleva años en continuo desarrollo,sin embargo, los matices de una lengua son dificiles y e spor esto que solo se ha desarrollado hasta cierto punto
	\end{flushleft}
	
	
	
	
	
    \section{Main milestones}
	\label{sec:examples}
	\section{Current status}
	\label{sec:examples}
	\section{Future challenges}
	\label{sec:examples}
	
	\section{Conclusion}
	\label{sec:examples}	
	
	
	
	%\section{}{Acknowledgements} \lipsum[6]
	
%	\newpage
	\bibliography{refs}
	
	\appendix
	
	
	
	\lipsum[7]
	
\end{document}