\documentclass[a4paper]{article}

\usepackage[pages=all, color=black, position={current page.south}, placement=bottom, scale=1, opacity=1, vshift=5mm]{background}
\SetBgContents{

}      % copyright

\usepackage[margin=1in]{geometry} % full-width

% AMS Packages
\usepackage{amsmath}
\usepackage{amsthm}
\usepackage{amssymb}

% Unicode
\usepackage[utf8]{inputenc}
\usepackage{hyperref}
\hypersetup{
	unicode,
%	colorlinks,
%	breaklinks,
%	urlcolor=cyan, 
%	linkcolor=blue, 
	pdfauthor={Author One, Author Two, Author Three},
	pdftitle={A simple article template},
	pdfsubject={A simple article template},
	pdfkeywords={article, template, simple},
	pdfproducer={LaTeX},
	pdfcreator={pdflatex}
}

% Vietnamese
%\usepackage{vntex}

% Natbib
\usepackage[sort&compress,numbers,square]{natbib}
\bibliographystyle{mplainnat}

% Theorem, Lemma, etc
\theoremstyle{plain}
\newtheorem{theorem}{Theorem}
\newtheorem{corollary}[theorem]{Corollary}
\newtheorem{lemma}[theorem]{Lemma}
\newtheorem{claim}{Claim}[theorem]
\newtheorem{axiom}[theorem]{Axiom}
\newtheorem{conjecture}[theorem]{Conjecture}
\newtheorem{fact}[theorem]{Fact}
\newtheorem{hypothesis}[theorem]{Hypothesis}
\newtheorem{assumption}[theorem]{Assumption}
\newtheorem{proposition}[theorem]{Proposition}
\newtheorem{criterion}[theorem]{Criterion}
\theoremstyle{definition}
\newtheorem{definition}[theorem]{Definition}
\newtheorem{example}[theorem]{Example}
\newtheorem{remark}[theorem]{Remark}
\newtheorem{problem}[theorem]{Problem}
\newtheorem{principle}[theorem]{Principle}

\usepackage{graphicx, color}
\graphicspath{{fig/}}

%\usepackage[linesnumbered,ruled,vlined,commentsnumbered]{algorithm2e} % use algorithm2e for typesetting algorithms
\usepackage{algorithm, algpseudocode} % use algorithm and algorithmicx for typesetting algorithms
\usepackage{mathrsfs} % for \mathscr command

\usepackage{lipsum}

% Author info
\pagebreak
\title{Translation}
\author{María Jesús Dueñas Recuero}

\date{

 \\ \texttt{MJesus.duenas1@alu.uclm.es}\\[2ex]%
%	\today
}
\setlength{\parindent}{4em}
\setlength{\parskip}{1em}
\begin{document}
	\maketitle
	\newpage
	\
	\tableofcontents
	\newpage
	\section{Introduction}
	\label{sec:intro}
	\begin{flushleft}
	Given the increase in globalisation, every day we are in contact with a greater number of cultures, which forces us, in a healthy way, to learn new languages.

	However, thanks to the development of technology and artificial intelligence, it is not necessarily necessary to study a language for years in order to understand it, given that translator websites and the translators themselves allow us to interact with it fluently.

	It is an opportunity to broaden our knowledge and understanding of other languages. As well as guaranteeing the correct use of the language for work and business, as they not only translate a language but also interpret emotions and feelings.
	
    \end{flushleft}
	
	


	
	
	\section{First aproaches}
	\label{sec:examples}
	\begin{flushleft}
	Today, machine translation is no longer a science fiction fantasy, it has been in continuous development for years, but the nuances of a language are difficult and that is why it has only been developed to a certain extent.\par
	
	In the early years around 1949, \textbf{Warren Weaver}, of the Rockefeller Foundation, set up a cryptography and language processing machine that was a precursor to the concept of machine translation. This project can be found in his \textbf{"Memorandum on Translation"} .\par

    In 1954 the founding research team of the Georgetown-IBM experiment carried out a demonstration of a machine that could translate 250 words from Russian into English.\par
    
    With this advance, people thought that machine translation would soon solve many problems and many translators began to fear for their jobs. However, the human brain is better able than a computer to access the complex framework of meaning, interpretation and syntax. In 1964, the Advisory Committee for Automatic Language Processing (ALPAC) declared that further investment in machine translation was no longer worthwhile.\par

    During the 1970s and 1990s Canada did not think like ALPAC, it developed the METEO system, which translated meteorological reports from English into French, capable of translating 80,000 words a day and was such a success that it was used until the first half of the 21st century.\par\par\par
    
    In addition, machine translation was used by the French textile industry institute to convert French summaries into English, German and Spanish. Around the same time, Xerox created its own system for translating technical manuals. At that time, machine translation was only used for the translation of technical documents.\par\par

    The 1980s saw the development of translation memory technology, which was the beginning of overcoming the challenge posed by the nuances of verbal communication, but the systems continued to face the same challenges in trying to convert a text into another language without losing its meaning.\par
    
    t was in 2000 when Franz JOsefOch, a computer scientist, won the machine translation contest that he became Google's Director of Translation Development. In 2012, Google announced its own translation application, Google translate. In addition, Japan is also leading the machine translation revolution with the creation of voice translation for mobile phones, which works in English, Japanese and Chinese. This is the result of the development of computer systems with a neural network model, rather than memory-based functions.\par
    
    That is why in 2016, Google informed the public that the application of a neural network technology improved the clarity of its translator, eliminating many of its errors. Called Google Neural Machine Translation (NMT), it began to translate language combinations that it had not been taught, for example, the programmers taught it to translate from English to Portuguese and Spanish, and it began to translate from Portuguese to Spanish without assigning it that combination.\par

	\end{flushleft}
	
	
	
	
	
    \section{Main milestones}
	\label{sec:examples}
	
	
	
	\subsection{European Association for Machine Translation (EAMT)}
	\begin{flushleft}
	The European Association for Machine Translation is made up of people interested in machine translation and translation software and includes users, developers and researchers, organises an annual congress at the European level, and every two years a joint congress with the IAMT, maintains a mailing list (mt-list@eamt.org) for the discussion of translation technology issues, and provides information on machine translation groups and research projects, and on events related to machine translation. On its website it provides information on research groups and projects and on events related to machine translation, and offers its members a free compendium of translation software, compiled by John Hutchins and updated annually.
	\end{flushleft}


	\subsection{Asia -Pacific Association for Machine Translation (AAMT)}
	\subsection{Association for Machine Translation inte AMericas (AMTA)}
	
	
	
	
	
	\section{Current status}
	\begin{flushleft}
	Machine translation research is very active internationally, both in terms of the number of groups involved and the activity in conferences, workshops and associations. The European Union maintains a system of funding for new projects related to language technologies and in particular to machine translation, allowing universities and companies to share knowledge and combine academic and commercial interests.\par

    The International Association for Machine Translation (IAMT), mentioned above, brings together three associations covering the European (European Association for Machine Translation), Asian (Asia-Pacific Association for Machine Translation) and American (Association for Machine Translation in the Americas) continents.\par

    The IAMT sponsors the <<Compedium of Translation Software>>, a list of translation software and resources organised by software type and vendor. It also organises a biannual conference on machine translation: the <<MT Summit>>, which brings together the three affiliated associations. It maintains and makes available on the Internet an electronic repository and an extensive bibliography of articles, books and papers on machine translation and related software.\par

    Research groups on machine translation are numerous. Sometimes it is a line of research within a research group covering different topics in natural language processing, sometimes machine translation is the main focus of research. Some research groups that stand out for their long-standing research or current projects include:
   
    \textbf{Pattern Recognition and Human Language Technology (PRHLT) research group, Polytechnic University of Valencia}
    \begin{flushleft}
    It investigates several areas, one of them being automatic translation of both text and spoken speech for restricted domains, based on statistics.the group has extensive experience in automatic speech recognition.the group has developed a large number of projects, a small sample of which are the following:
     \begin{itemize}
    \item Development of Statistical Techniques for Adaptive and Interactive Learning in Computer-Assisted Translation.
    \item Translation support based on translation memories.
    \item \textbf{SISHITRA:} hybrid systems for Valencian-Spanish translation from speech and text.
    \item \textbf{TRACOM: }Translation and Understanding of Spoken Language through Exemplar-based Learning Techniques.
    \end{itemize}
    \end{flushleft}
    
    \textbf{TALP, Centre for Language and Speech Technologies and Applications of the Polytechnic University of Cataluña}
    \begin{flushleft}
    They are developing their own statistical speech translation system, incorporating linguistic knowledge (morphological, syntactic and semantic) to improve its performance. The working languages are Spanish, Catalan and English, but also Madarin and Arabic. All this, in the framework of national and European projects. They are currently participating in the European project \textbf{MOLTO}, for a tool to translate texts between multiple languages instantly and with high quality, based on interlingua. Projects to be highlighted:
    \begin{itemize}
        \item \textbf{EuroOpenTrad: }open source machine translation for the European integration of the languages of the Spanish State. The translation system is available at http://www.opentrad.org/.
        \item\textbf{OpenTrad: }open source machine translation for the languages of the Spanish State.
        \item \textbf{MOLTO: }Multilingual Online Translation, funded by the European Union.
    \end{itemize}

    \end{flushleft}
    \subsection{Deepl vs Google translate}
\end{flushleft}

	
	\label{sec:examples}
	\section{Future challenges}
	\label{sec:examples}
	
	\section{Conclusion}
	\label{sec:examples}	
	
	
	
	%\section{}{Acknowledgements} \lipsum[6]
	
%	\newpage
    \usepackage{biblatex}
    \section{Bibliography}
    @online{knuthwebsite,
    author    = "Hello yuqo",
    title     = "Nacimiento e historia de la traduccion automatica",
    url       = "https://www.yuqo.es/nacimiento-e-historia-de-la-traduccion-automatica/",
    keywords  = "latex,knuth"
    
    https://cvc.cervantes.es/lengua/anuario/anuario_10-11/alcina/p05.htm
}
	
\end{document}